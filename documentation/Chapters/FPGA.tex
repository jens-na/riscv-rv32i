\chapter{FPGA} % Main chapter title
\label{FPGA} % For referencing the chapter elsewhere, use \ref{Chapter1} 

\section{Einleitung}
Ein FPGA (Field Programmable Gate Array) ist ein integrierter Schaltkreis, in den eine beliebige logische Schaltung programmiert werden kann.

Der Vorteil für das vorliegende Projekt besteht insbesondere darin, dass FPGA's beliebig oft umprogrammiert werden können. Sie eignen sich dadurch für die Vorbereitungs- und Testphase solcher Hardwareprojekte, da sie kostengünstig und verhältnismäßig schnell neu konfiguriert werden können.

\section{Technik}
\paragraph{Logikblöcke.} FPGA's bestehen aus einem zweidimensionalen Array von programmierbaren Blöcken, den \emph{Logic Cells}. Diese bestehen aus mindestens einem Look-Up Table (LUT) und einem Flipflop. Look-Up Tables haben üblicherweise 4-6 Eingänge und sind so konfigurierbar, dass sie eine beliebige Binärfunktion umsetzen.

\paragraph{Verbindungen.} Diese Logikblöcke sind wiederum mit konfigurierbaren Verbindungen verbunden. Dafür liegt zwischen den Blöcken ein Gitter aus Leitungen, an deren Kreuzungspunkten die Signalverteilung beliebig geschaltet werden kann.

\section{Verwendeter FPGA}
Für dieses Projekt wurde das \emph{Nexys 4}-Board mit einem Xilinx \emph{Artix-7}-FPGA verwendet. Dieser enthält $101440$ Logic Cells, die in $15850$ Logic Slices organisiert sind. Diese sind wiederum aus jeweils vier LUT's mit sechs Inputs und acht Flipflops aufgebaut. Der FPGA erlaubt die Implementierung von bis zu $1188$ kB Distributed RAM (durch die Logikbausteine implementierter RAM). Das Board wird mit einem internen 100Mhz Takt versorgt. \cite{Artix}