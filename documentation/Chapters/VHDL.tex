\chapter{VHDL} % Main chapter title
\label{VHDL} % For referencing the chapter elsewhere, use \ref{Chapter1} 

%----------------------------------------------------------------------------------------
\section{Einleitung}
VHDL (\textit{Very High Speed Integrated Circuit Hardware Description Language}) ist eine Hardwarebeschreibungssprache. Mit ihr kann das Verhalten einer logischen Schaltung textuell beschrieben werden. Auf die konkrete Umsetzung der beschriebenen Schaltung hat der Anwender dabei jedoch nur bedingten Einfluss. Anstatt eine Schaltung aus ihren einzelnen elektronischen Teilen zu entwerfen, funktioniert VHDL auf einer höheren Abstraktionsschicht. Es wird lediglich beschrieben, wie sich eine Schaltung oder ein logischer Teil einer Schaltung in Abhängigkeit bestimmter Eingangsgrößen verhält.

Das beschriebene Verhalten kann dann mittels entsprechender weiterführender Tools \textit{simuliert} und in eine echte Schaltung \textit{synthetisiert} und anschließend bspw. auf einem FPGA \textit{implementiert} werden.
%----------------------------------------------------------------------------------------

%----------------------------------------------------------------------------------------
\section{Syntax?}
%----------------------------------------------------------------------------------------

%----------------------------------------------------------------------------------------
\section{Unterschiede zur Programmiersprache / Herausforderungen}
%----------------------------------------------------------------------------------------

%----------------------------------------------------------------------------------------
\section{Simulation und Synthese}
%----------------------------------------------------------------------------------------
