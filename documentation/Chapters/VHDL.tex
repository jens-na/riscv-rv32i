\chapter{VHDL} % Main chapter title
\label{VHDL} % For referencing the chapter elsewhere, use \ref{Chapter1} 

%----------------------------------------------------------------------------------------
\section{Einleitung}
VHDL (\textit{Very High Speed Integrated Circuit Hardware Description Language}) ist eine Hardwarebeschreibungssprache, die 1980 entwickelt wurde. Mit ihr kann das Verhalten einer logischen Schaltung formalisiert textuell beschrieben werden. 

Anstatt eine Schaltung aus ihren einzelnen elektronischen Teilen zu entwerfen, arbeitet VHDL auf einer höheren Abstraktionsschicht. Es wird lediglich beschrieben, wie sich eine Schaltung oder ein logischer Teil einer Schaltung in Abhängigkeit bestimmter Eingangsgrößen verhält.
%----------------------------------------------------------------------------------------

%----------------------------------------------------------------------------------------
\section{Entwicklung mit VHDL}
\paragraph{Entities.} In VHDL werden \textit{Entities} definiert, die eine Klasse von Schaltelementen repräsentieren. Entities bestehen üblicherweise aus mehreren Ein-und Ausgangs\textit{ports}. Innerhalb der \textit{Architecture} einer \textit{Entity} wird festgelegt, wie sich die Ausgangssignale in Abhängigkeit der Eingangssignale verhalten.

\paragraph{Parallelität.} Bei der Entwicklung ist insbesondere zu beachten, dass das Verhalten einer Entity, anders als von Programmiersprachen gewohnt, im Allgemeinen nicht sequentiell abgearbeitet wird. Vielmehr findet die Signalverarbeitung parallel statt.

\paragraph{Processes.} Innerhalb des Verhaltens können deshalb Prozesse deklariert werden, deren Instruktionen sequentiell abgearbeitet werden. Zu diesem Zweck können innerhalb eines Prozesses auch \textit{Variablen} verwendet werden.

\paragraph{Strukturmodelle.} Neben Verhaltensmodellen kann mit Strukturmodellen eine Schaltung aus mehreren Entities beschrieben werden. Dafür werden sog. \textit{Components} (Instanzen von zuvor beschriebenen Entities) erstellt und deren Ports miteinander verbunden.

So können verhältnismäßig einfach komplexe Verschaltungen beschrieben werden.
%----------------------------------------------------------------------------------------
 
%----------------------------------------------------------------------------------------
\section{Umsetzung von VHDL}
Auf die konkrete Umsetzung der beschriebenen Schaltung hat der Anwender dabei nur bedingten Einfluss, da die Implementierung der Schaltung in Hardware selbst üblicherweise von automatisierten Werkzeugen durchgeführt wird. Dabei lassen sich mehrere Schritte unterscheiden. In diesem Projekt wurde die Entwicklungsumgebung Vivado Design Suite von Xilinx verwendet, um den VHDL-Code umzusetzen.

\paragraph{Simulation.} In einem ersten Schritt kann der VHDL Code \emph{simuliert} werden, was vor allem in der Testphase von Bedeutung ist. Dabei wird der Code noch nicht auf der Hardware ausgeführt, sondern das Verhalten der Schaltung mittels sogenannter Testbenches simuliert, die die Schaltung mit virtuellen Eingangssignalen (z. B. Taktrate) versorgen. 

TODO: Screenshot?

\paragraph{Synthese.} Das simulierte Verhalten kann anschließend \emph{synthetisiert} werden. Dabei ist zu beachten, dass nicht die vollständige Menge von zulässigem (und simulationsfähigem) VHDL-Code letztlich in eine physische Schaltung synthetisierbar ist (so z. B. \emph{wait}-Statements). Das ist abhängig von der verwendeten Zieltechnologie (FPGA, ASIC) und dem verwendeten Synthesetool. Das Ergebnis der Synthese ist eine Netzliste, die einer textuellen Beschreibung des umgesetzten Schaltplans entspricht.

\paragraph{Implementierung.} Diese Netzliste ist Grundlage der Hardwareimplementierung. Dabei wird innerhalb des \emph{place and route} Vorgangs unter Berücksichtigung der verfügbaren Hardware und bestimmter zeitlicher Beschränkungen berechnet, wo die einzelnen Schaltelemente auf der Hardware implementiert werden und wie sie verbunden werden.
%----------------------------------------------------------------------------------------
