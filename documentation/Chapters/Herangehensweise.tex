\chapter{Organisation} 
\label{organisation} 

\section{Teammanagement}
Um das gemeinsame Ziel, nämlich die Implementierung einer RISC-V CPU,
umzusetzen musste vor Beginn der eigentlichen Arbeit gewisse
Vorkehrungen getroffen werden, um die Arbeit im Team zu ermöglichen.
Hierzu gehört insbesondere die Einigung auf eine bestimmte
Projektstruktur und eine Versionsverwaltung in welcher die Quelldateien
der Implementierung abgelegt werden können. Einige dieser
organisatorischen Angelegenheiten werden in den folgenden Kapiteln 
genauer erläutert. 

\subsection{Projektstruktur}
Wie bereits in Kapitel \ref{vivado} erwähnt wurde ist als
Entwicklungsumgebung Vivado zum Einsatz gekommen. Vivado gibt für
VHDL-Projekte bereits eine bestimmte Projektstruktur vor, diese ist
jedoch für die Arbeit im Team ungeeignet, da sie Dateien mit festen 
Pfadangaben  und vielen temporären, automatisch erzeugten Dateien
beinhaltet. Dadurch können Vivado-Projekte nicht ohne Weiteres in einer
gemeinsamen Versionsverwaltung geteilt werden. Um trotz allem eine
Projektstruktur zu erstellen, die auch für das Versionskontrollsystem
\emph{git} geeignet ist, sind verschiedene Anpassungen nötig gewesen.

\paragraph{Build-Script.}  
Vivado selbst ist in der Programmiersprache Tcl geschrieben und kann
ebenfalls mit dieser Programmiersprache erweitert werden. Es ist möglich
ein Tcl-Script zu erstellen, was ein temporäres Vivado-Projekt erstellt
in dem gearbeitet werden kann. Der Vorteil ist, dass die Quelldateien in
einem anderen Ordner aufbewahrt werden und unnötige Dateien des
temporären Vivado-Projektes nicht unter Versionskontrolle gestellt werden.

\paragraph{Aufbau.} Es ergibt sich folgende Projektstruktur. Alle Ordner
bis auf \code{work/} werden in einem Versionskontrollsystem abgelegt:
\begin{lstlisting}[inputencoding={utf8},extendedchars=false,escapeinside=``]
riscv/
  documentation/ -- Verzeichnis f`ü`r Dokumentation
  script/        -- Verschiedene Hilfsprogramme 
  src/           -- VHDL Quelldateien 
  test/          -- VHDL Testbench Dateien 
  work/          -- tempor`ä`res Vivado-Projekt (nicht in git) 
  build.sh       -- Programm zur Generierung des Vivado-Projekts
  build.tcl      -- Tcl Script zur Generierung des Vivado-Projekts
  start.sh       -- Start-Programm f`ü`r Vivado 
\end{lstlisting}


\subsection{Versionskontrolle}
Als Versionsverwaltungssystem hat sich die Gruppe für \emph{git}
entschieden. Die Hochschule RheinMain bietet für Softwareprojekte
zusätzlich einen \emph{Gitlab}-Server an, auf dem das Projekt unter
dem Namen
\textbf{vhdl-cpu/riscv}\footnote{\url{https://zenon.cs.hs-rm.de/vhdl-cpu/riscv}
(08.03.2017)} geführt ist.

\subsection{Arbeitsweise}

\paragraph{Workflow.} Das Projekt wurde unter Berücksichtigung des sogenannten
\emph{GitHub-Flows}\footnote{\url{https://guides.github.com/introduction/flow/}
(08.03.2017)}
entwickelt. Das bedeutet, dass verschiedene Regeln, was das 
Versionskontrollsystem betrifft, eingehalten wurden:

\begin{enumerate}
\item Es gibt einen Branch \emph{master}, der zu jeder Zeit kompilierbar
und in Vivado simulierbar ist.
\item Entwicklungen werden in neuen Branches durchgeführt (sogenannte
Feature-Branches).
\item Ist die Entwicklung eines Features für das Projekt abgeschlossen
wird ein \emph{Merge-Request} des entsprechenden Entwicklers gestellt.
\item Die Gruppe entscheidet anschließend gemeinsam, ob das Feature in
den \emph{master}-Branch übernommen werden soll oder, ob noch Änderungen
zur Fertigstellung notwendig sind.
\end{enumerate}

Die Entwicklung nach diesem Prinzip hat mehrere Vorteile. Es ist zum
Beispiel direkt nachvollziehbar wer, welche Teile der Entwicklung
übernommen hat. Auch das gemeinsame Entscheiden über die Integration in
den \emph{master}-Branch kann helfen Fehler frühzeitig zu erkennen.

Zur Synthese der RISC-V-CPU auf dem FPGA wurde zusätzlich noch ein Branch
namens \emph{master\_\,fpga} angelegt. Dieser Branch enthält zusätzlich zu
der CPU noch FPGA-spezifische Änderungen, wie zum Beispiel die
Constraint-Datei für die Portbelegung.

\paragraph{Arbeitsteilung.} In Kapitel \ref{sec:komponenten} wurden die
verschiedenen Komponenten der CPU vorgestellt. Die Entwicklung der
einzelnen Bausteine wurde in der Gruppe aufgeteilt. Anschließend mussten
die verschiedenen Maschinenbefehle, wie in Tabelle
\ref{maschinenbefehle} zu sehen, umgesetzt werden. Diese wurden
nach der Entwicklung der Komponenten aufgeteilt und in Feature-Branches 
entwickelt. Zum Beispiel gibt es einen Branch namens \emph{ar\_\,load}, der die 
Entwicklung der Load/Store-Befehle wie \code{lw, lb, lh} beinhaltet.

\section{Testumgebung} 
\subsection{Testbenches}
Zu jeder erstellten Komponente, die in Kapitel \ref{sec:komponente}
erläutert wurde sind Testbenches erstellt worden. Eine Testbench testet
\emph{eine} Komponente unabhängig und stellt somit einen Modultest der
entsprechenden Komponente dar. Zusätzlich wurde eine Testbench namens
\code{main\_tb} erzeugt, die das Zusammenspiel der einzelnen Komponenten
testet (Integrationstest). In der Testbench \code{main\_tb} kann somit
die Arbeitsweise der CPU simuliert und untersucht werden. 

Generell wurden Testbenches während der Entwicklung verwendet, um zu
überprüfen, ob sich die Signale und Funktionsweise der Komponenten
richtig verhalten.
\subsection{White-Box-Test}
Zusätzlich wurden mehrere White-Box-Tests durchgeführt. Bei dieser Art
von Test sind interne Details der Implementierung bekannt und werden
überprüft\cite[S. 311]{Kleuker}. In der vorliegenden Implementierung wird getestet, ob die CPU 
die gewünschten Änderungen an den  Registern nach Abarbeitung einer 
Instruktion durchführt.

\begin{table}[h]
\centering
\begin{tabular}{|l|l|l|l|}
Instruktion & x5(t0) & x6(t1) & x7(t2) \cr
\hline
lhu t1,506(zero) & 0x81 & 0x8713 & 0x82070785 \cr
add t2, t0, t1 & 0x81 & 0x8713 & 0x8794 \cr 
sub t2,t0,t1 & 0x81 & 0x8713 & 0xffff796e \cr 
xor t2,t0,t1 & 0x81 & 0x8713 & 0x8792 \cr
\dots & \dots & \dots & \dots
\end{tabular}
\caption{Ausschnitt der White-Box-Tests und die Beschreibung der
Register nachdem der entsprechende Test durchgeführt wurde.}
\label{tab:whiteboxtest}
\end{table}
Die Testfälle wurde manuell in der Simulationsansicht von Vivado überprüft.

\subsection{Black-Box-Test}
Nachdem die Implementierung abgeschlossen war und es möglich war den
vorliegenden FPGA mit der RISC-V-CPU zu „programmieren“ wurden
Black-Box-Tests mit dem FPGA durchgeführt. Bei einem solchen Test wird
nur das „äußere Verhalten des Systems“ betrachtet \cite[S. 311]{Kleuker}. Dafür wurden die niederwertigsten 16 Bits
des Registers \code{a0} (Status-Register) mit den LEDs auf dem FPGA verbunden. Es
wurden verschiedene Programme ausgeführt und die Ausgabe der LED-Anzeige
überprüft.

\subsection{Hilfsprogramm \emph{riscv-instr}}
Um gültige 32 Bit Maschinenbefehle für die CPU testweise zu erstellen
wurde ein Hilfsprogramm namens \emph{riscv-instr} erstellt, welches in
der Lage ist zu einem gültigen Assemblerprogramm die entsprechenden
Instruktionen für die CPU auszugeben. Mit diesem Programm ist es somit
möglich Instruktionen zu erzeugen, die in den RAM der CPU geladen werden
können.

Beispiel-Assembler-Programm:
\begin{lstlisting}
  .globl main
  .type main, @function
main:
  addi x1,x2,12 
  add  x1,x1,x2
\end{lstlisting}

Nach Programmaufruf werden die beiden Instruktionen als Hexadezimal
ausgegeben:
\begin{lstlisting}[language=bash]
riscv-instr testasm.s
0x00c10093 
0x002080b3 
\end{lstlisting}
Die beiden Instruktionen können in den RAM der CPU geladen werden und
können von der RISC-V CPU verarbeitet werden. Das Programm
\emph{riscv-instr} benötigt zur Erzeugung der Maschinenbefehle die 
entsprechende GNU Cross Compiler Toolchain für RISC-V
\footnote{\url{https://github.com/riscv/riscv-gnu-toolchain}
(08.03.2017)}. 

\subsection{RAM-Initialisierung}
Um bei der Erstellung des RAMs möglichst flexibel zu sein kann der
Inhalt aus einer Datei eingelesen werden. \cite[S. 187]{XST}
Hierzu müssen alle Instruktion, die in den RAM der CPU geladen werden
sollen, als 32-Bit-Vektor in einer Datei abgelegt werden:

Beispiel-Datei:
\begin{lstlisting}
00000000000000000000001010010011
00000001001000101110010100010011
00000000000001010000000001100111
...
\end{lstlisting}

Die Funktion \ref{src:blockram_read_file} liest beim Synthese-Vorgang
den Inhalt der entsprechenden Datei und füllt den RAM mit den
Instruktionen. Dies war besonders für den Black-Box-Test notwendig.
