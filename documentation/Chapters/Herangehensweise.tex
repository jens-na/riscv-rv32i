\chapter{Herangehensweise und Organisation} 
\label{Herangehensweise} 

\section{Teammanagement}
Um das gemeinsame Ziel, nämlich die Implementierung einer RISC-V CPU,
umzusetzen musste vor Beginn der eigentlichen Arbeit gewisse
Vorkehrungen getroffen werden, um die Arbeit im Team zu ermöglichen.
Hierzu gehört insbesondere die Einigung auf eine bestimmte
Projektstruktur und eine Versionsverwaltung in welcher die Quelldateien
der Implementierung abgelegt werden können. Einige dieser
organisatorischen Angelegenheiten werden in den folgenden Kapiteln 
genauer erläutert. 

\subsection{Projektstruktur}
Wie bereits in Kapitel \ref{vivado} erwähnt wurde ist als
Entwicklungsumgebung Vivado zum Einsatz gekommen. Vivado gibt für
VHDL-Projekte bereits eine bestimmte Projektstruktur vor, diese ist
jedoch für die Arbeit im Team ungeeignet, da sie Dateien mit festen 
Pfadangaben  und vielen temporären, automatisch erzeugten Dateien
beinhaltet. Dadurch können Vivado-Projekte nicht ohne Weiteres in einer
gemeinsamen Versionsverwaltung geteilt werden. Um trotz allem eine
Projektstruktur zu erstellen, die auch für das Versionskontrollsystem
„git“ geeignet ist, sind verschiedene Anpassungen nötig gewesen.

\paragraph{Build-Script.}  
Vivado selbst ist in der Programmiersprache Tcl geschrieben und kann
ebenfalls mit dieser Programmiersprache erweitert werden. Es ist möglich  
\paragraph{Aufbau.}  

\subsection{Versionskontrolle}
\subsection{Arbeitsteilung}

\section{Testumgebung}
\subsection{Excel}
\subsection{risc-v-instr}
\subsection{Hardware Tests}
- Constraints/LEDs
