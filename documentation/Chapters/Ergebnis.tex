\chapter{Ergebnis} % Main chapter title
\label{Ergebnis} % For referencing the chapter elsewhere, use \ref{Chapter1} 

%----------------------------------------------------------------------------------------
\section{Kennzahlen}
Die Leistung einer CPU wird in der Regel durch folgende Kennzahlen
dargestellt \cite[S. 43]{Hennessy}:
\begin{itemize}
    \item Schaltfrequenz in MHz \emph{(clock rate)}
    \item Anzahl der CPU-Kerne \emph{(cpu cores)}
    \item Instruktionen pro Clock-Zyklus \emph{(instruction per cycle)} 
\end{itemize}

\paragraph{Schaltfrequenz.} Mit der Entwicklungsumgebung Vivado ist es
möglich herauszufinden wie hoch die Schwaltfrequenz der Clock sein kann,
damit die Implementierung der RISC-V-CPU auf dem vorliegenden FPGA
lauffähig ist. Hierzu sind verschiedene Werte als „Clock Constraint“
getestet worden. Nach der Synthese kann man im Synthese-Report von
Vivado folgenden Ausschnitt beobachten: 
\begin{lstlisting}
Max Delay Paths
---
Slack (MET) :      1.018ns  (required time - arrival time)
  Source:          c_pc/cnt_reg_reg[2]_rep__3/C
  Destination:     c_registerfile/reg_blocks_reg[10][10]/D
  Path Group:      m_clk
  Requirement:     21.277ns  
  Data Path Delay: 20.123ns  
\end{lstlisting}
Als Clock-Constraint wurde $47 MHz \equiv 21.277ns$ angegeben.
Das bedeutet, dass zwischen zwei steigenen Flanken $21.277ns$ vergeht,
dies entspricht einer Schaltfrequenz von $47 MHz$.

Der Synthese-Report zeigt, dass die größte Verzögerung im Design
$20.123ns$ beträgt. Dadurch ist das Design mit $47 MHz$ ($21.277ns$) lauffähig 
.
Bereits bei einer „Clock Constraint“ mit $48 MHz$ wird die Anforderung
überschritten.

Das bedeutet, dass die Implementierung der RISC-V-CPU mit einer Clock
von maximal $\mathbf{47 MHz}$ betrieben werden kann.

\paragraph{CPU-Kerne.} Die umgesetzte CPU besitzt \textbf{einen} Kern.
Eine Parallelausführung von Programmen ist somit nicht möglich.

\paragraph{Instruktionen.} Eine weitere Kennzahl, die die Leistung einer
CPU beschreibt sind die Instruktionen, die pro Clock-Zyklus ausgeführt
werden (IPC). 
In der vorliegenden Implementierung der RISC-V-CPU beträgt \textbf{IPC =
1/2}. Es werden also pro Maschineninstruktion genau \textbf{zwei}
Clock-Zyklen benötigt. Grund hierfür sind die Schreibvorgänge in RAM und
Register. Weitere Informationen hierfür sind in Kapitel
\ref{sec:ram_register_schreibvorgaenge} zu finden.










%----------------------------------------------------------------------------------------
