\chapter{Einleitung} % Main chapter title
\label{Einleitung} % For referencing the chapter elsewhere, use \ref{Chapter1} 

%----------------------------------------------------------------------------------------

% Define some commands to keep the formatting separated from the content 
\newcommand{\keyword}[1]{\textbf{#1}}
\newcommand{\tabhead}[1]{\textbf{#1}}
\newcommand{\code}[1]{\texttt{#1}}
\newcommand{\file}[1]{\texttt{\bfseries#1}}
\newcommand{\option}[1]{\texttt{\itshape#1}}

%----------------------------------------------------------------------------------------
\section{Zusammenfassung}
Diese Arbeit wurde im Rahmen des Wahlprojekts \textit{Software- Hardwareschnittstellen} bei Prof. Dr. Reith an der Hochschule RheinMain verfasst. Sie dokumentiert die Konzeptionalisierung und die Implementierung einer CPU auf einem \textit{FPGA} Board. Die verwendete Befehlssatzarchitektur basiert auf dem offenen \textit{\href{https://riscv.org/}{RISC-V}} Design. Die Architektur wurde unter Verwendung der Hardwarebeschreibungssprache \textit{VHDL} implementiert. Anschließend wurde sie testweise für das \textit{\href{https://reference.digilentinc.com/reference/programmable-logic/nexys-4/start}{Xilinx Nexys 4\textsuperscript{\textregistered}}}
 Board mit Artix-7\textsuperscript{\textregistered} FPGA synthetisiert.\\
%----------------------------------------------------------------------------------------
 
%----------------------------------------------------------------------------------------
\section{Aufgabenstellung}
Die Aufgabenstellung beinhaltet die Umsetzung einer eigenen CPU- ... \cite{Ashenden:609207}

\subsection{...}
%----------------------------------------------------------------------------------------