\chapter{Einleitung} % Main chapter title
\label{Einleitung} % For referencing the chapter elsewhere, use \ref{Chapter1} 

%----------------------------------------------------------------------------------------

% Define some commands to keep the formatting separated from the content 
\newcommand{\keyword}[1]{\textbf{#1}}
\newcommand{\tabhead}[1]{\textbf{#1}}
\newcommand{\code}[1]{\texttt{#1}}
\newcommand{\file}[1]{\texttt{\bfseries#1}}
\newcommand{\option}[1]{\itshape#1}

Diese Arbeit wurde im Rahmen des Wahlprojekts \textit{Software- Hardwareschnittstellen} bei Prof. Dr. \emph{Reith} an der Hochschule RheinMain verfasst. Die vorgegebene Projektaufgabe beinhaltet die Implementierung einer CPU auf einem \emph{FPGA} mittels der Hardwarebeschreibungssprache \emph{VHDL}. Die CPU soll die offene \emph{RISV-V} Architektur umsetzen. Diese Dokumentation beschreibt das Konzept, die Planung und die Umsetzung dieser Aufgabenstellung.

Zunächst werden hierfür einige allgemeine Grundbegriffe, die zum Verständnis des Projekts erforderlich sind, kurz erläutert: Die FPGA-Technologie, die Hardwarebeschreibungssprache VHDL, die offene Befehlssatzarchitektur RISC-V und welche Teilmenge des (modularen) RISC-V Befehlssatzes umgesetzt wurde. Dabei soll jedoch lediglich ein kurzer Überblick über die Materie verschafft werden. Für das tiefere Verständnis und Anleitungen sei auf die entsprechende Fachliteratur für VHDL\citep{Ashenden:609207}, FPGA-Programmierung\cite{Chu} mit Vivado\cite{churiwala} und auf die RISC-V Spezifikation\citep{RISC} verwiesen.

Der Schwerpunkt liegt stattdessen auf der Beschreibung der Umsetzung der gesetzten Aufgabenstellung. Dafür wurde in Kapitel \ref{Umsetzung} die konkrete Implementierung der einzelnen Komponenten der RISC-V CPU in VHDL dokumentiert. Anschließend stellt Kapitel \ref{organisation} die Herausforderungen des Team- und Projektmanangements dar und wie mit ihnen umgegangen wurde. Es wird die Arbeitsteilung, die Versionskontrolle mit \emph{git} und die Qualitätssicherung durch Softwaretests beschrieben.

Kapitel \ref{Probleme} dokumentiert die Probleme, auf die das Team im Laufe der Entwicklung gestoßen ist und inwiefern Teile der Umsetzung limitiert oder eingeschränkt werden mussten. 

Schließlich werden in Kapitel \ref{Ergebnis} die Ergebnisse der Arbeit dargestellt und die Details und Kennwerte der umgesetzten CPU zusammengefasst. Es wird außerdem ein kurzer Ausblick auf künftige Entwicklungen des Projekts gegeben.

Das Ergebnis dieser Arbeit ist unter \ref{???} frei verfügbar.
